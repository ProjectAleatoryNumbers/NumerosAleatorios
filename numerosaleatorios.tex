\documentclass[conference]{IEEEtran}
\IEEEoverridecommandlockouts
\usepackage{cite}
\usepackage{amsmath,amssymb,amsfonts}
\usepackage{algorithmic}
\usepackage{graphicx}
\usepackage{textcomp}
\usepackage{xcolor}
\def\BibTeX{{\rm B\kern-.05em{\sc i\kern-.025em b}\kern-.08em
    T\kern-.1667em\lower.7ex\hbox{E}\kern-.125emX}}
\begin{document}

\title{Generaci\'on de n\'umeros aleatorios con distribuci\'on normal\\
\IEEEauthorblockA{\\{*}Castillo Flores Junior Manuel\\{*}Calapuja Apaza Luis Alberto}\\
\centerline{\textit{Facultad de Ciencias, Universidad Nacional de Ingenier\'ia}}\\
\centerline{\textit{Lima-Per\'u}}\\
\author{\IEEEauthorblockN{\IEEEauthorblockA{\textbf{juniorcastillon6$@$gmail.com\\lcalapujaa$@$uni.pe}}}}
}
\maketitle
\begin{abstract}
%%aqui va el abstract

\end{abstract}
\begin{IEEEkeywords}
%%palabras claves 
Monte Carlo, congruencia lineal, pseudo-aleatoriedad , estoc\'astico, 
\end{IEEEkeywords}

\section{Introducci\'on}
Antes de la aparici\'on de las computadoras y su capacidad de c\'alculo en procesos estoc\'asticos, existieron diferentes m\'etodos para la generaci\'on de n\'umeros aleatorios, estos estaban basados en procedimientos mec\'anicos que generaban enormes tablas de n\'umeros (Tippett, Kendall y Babbington, Rand Corporation, etc). M\'as adelante al rededor de los an\~os 50 del siglo pasado aparecen t\'ecnicas matem\'aticas para la simulaci\'on de variables aleatorias y procesos no deterministas, dicho m\'etodo lleva por nombre Monte Carlo.\\
Por otro lado existen dos tipos de generadores para n\'umeros aleatorios, los que se basan en fen\'omenos f\'isicos tales como el ruido atmosf\'erico, que tiene un alto grado de entrop\'ia (ya que no se conocen las condiciones iniciales que generan estos ruidos), estos son denominados generadores de verdaderos n\'umeros aleatorios (TRNG),en contrariedad con este m\'etodo los generadores pseudoaleatorios (PRNG) \cite{b1}, necesitan de un estado inicial para generar los n\'umeros mediante una secuencia algor\'itmica, por esto carecen de entrop\'ia, es decir, si se conoce el estado inicial es muy probable que se pueda recalcular una secuencia de n\'umeros, es por esto que un buen generador pseudoaleatorio debe de incorporar alg\'un grado de entrop\'ia dentro de su ecuaci\'on,por lo tanto este tipo de t\'ecnica debe de incluir alguna complejidad adicional.\\
En este proyecto utilizaremos m\'etodos de congruencia lineal y no lineal\cite{b2} para la generaci\'on de nuestros n\'umeros pseudoaleatorios y compararemos los datos obtenidos para cada uno de estos algoritmos.
\section{Estado del arte}
En el art\'iculo Random Number Generation: Types and Techniques \cite{b2} se describe dos m\'etodos de generaci\'on de n\'umeros aleatorios, la primera se basa en tomar como modelo fenomenos del entorno fisico cuyos patrones no son aleatorios, hay quienes discrepan sobre esta aleatoriedad, ya que para que se d\'e un fen\'omeno debe de existir efectos que lo causen, si se llegara a conocer las condiciones iniciales que causan estos fen\'omenos, estos dejar\'ian de ser no deterministas, sin embargo, esto es casi imposible ya que los factores causantes de un fen\'omeno son pr\'acticamente infinitos, esto se puede describir como un efecto mariposa, es decir, que si se produjera una peque\~na perturbaci\'on en las condiciones iniciales los cambios en los fen\'omenos de la naturaleza ser\'ian enormes. Entre los fen\'omenos f\'isicos tomados como fuente de aleatoriedad tenemos: ruido atmosf\'erico, decaimiento radiactivo, lasers y circuitos osciladores.\\
Por otro lado tambi\'en existen generadores que no dependen de fen\'omenos de la realidad, a estos se los conoce como generadores pseudoaleatorios, estos generadores en principio parecen producir secuencias aleatorias para cualquiera que no conozca el valor inicial secreto. En un generador pseudoaleatorio b\'asico, el valor inicial es el \'unico factor en que se introduce la entrop\'ia al sistema. A diferencia de los verdaderos generadores de n\'umeros aleatorios (que toman la entrop\'ia de un fen\'omeno y lo transforman en n\'umeros), un generador pseudoaleatorio necesita encontrar alguna entrop\'a para mantenerse impredecible. Las t\'acticas cl\'asicas para lograr esto incluyen tomar la hora del d\'ia, la ubicaci\'on del mouse o la actividad en el teclado, esto no nos asegura que alguien no pueda replicar una secuencia conociendo los valores iniciales, ya que un humano puede replicar estas t\'acticas. Un generador pseudoaleatorio confiable es aquel que no nos permite replicar con facilidad una secuencia de n\'umeros, este debe de tener variables intermedias que no puedan determinarse con facilidad.

El m\'etodo de congruencia lineal de Lehmer \cite{b3} es uno de los m\'etodos m\'as conocidos para la generaci\'on de n\'umeros pseudoaleatorios, consiste en escoger convenientemente una semilla $X{0}$, un multiplicador  

\section{Dise\~no del experimento}
\begin{itemize}

\item 
\end{itemize}

modificando el archivo

\section*{Acknowledgment}



\begin{thebibliography}{00}
\bibitem{b1} David DiCarlo (2012), Random Number Generation: Types and Techniques, [archivo PDF]. Disponible en: https://pdfs.semanticscholar.org
\bibitem{b2} James E. Gentle, Random Number Generation and Monte Carlo Methods, 2da ed., Estados Unidos, 2005, pp.11--38.
\bibitem{b3}
\end{thebibliography}
\vspace{12pt}

\end{document}
