\documentclass[conference]{IEEEtran}
\IEEEoverridecommandlockouts
\usepackage{cite}
\usepackage{amsmath,amssymb,amsfonts}
\usepackage{algorithmic}
\usepackage{graphicx}
\usepackage{textcomp}
\usepackage{xcolor}
\def\BibTeX{{\rm B\kern-.05em{\sc i\kern-.025em b}\kern-.08em
    T\kern-.1667em\lower.7ex\hbox{E}\kern-.125emX}}
\begin{document}

\title{Generaci\'on de n\'umeros aleatorios con distribuci\'on normal\\
\IEEEauthorblockA{\\{*}Castillo Flores Junior Manuel, }\\
\centerline{\textit{Facultad de Ciencias, Universidad Nacional de Ingenier\'ia}}\\
\centerline{\textit{Lima-Per\'u}}\\
\author{\IEEEauthorblockN{\IEEEauthorblockA{\textbf{juniorcastillon6$@$gmail.com}}}}
}
\maketitle
\begin{abstract}
%%aqui va el abstract
\end{abstract}
\begin{IEEEkeywords}
%%palabras claves
\end{IEEEkeywords}

\section{Introducci\'on}
Antes de la aparici\'on de las computadoras y su capacidad de c\'alculo en procesos estoc\'asticos, existieron diferentes m\'etodos para la generaci\'on de n\'umeros aleatorios, estos estaban basados en procedimientos mec\'anicos que generaban enormes tablas de n\'umeros (Tippett, Kendall y Babbington, Rand Corporation, etc). M\'as adelante al rededor de los an\~os 50 del siglo pasado aparecen t\'ecnicas matem\'aticas para la simulaci\'on de variables aleatorias y procesos no deterministas, dicho m\'etodo lleva por nombre Monte Carlo.\\
Por otro lado existen dos tipos de generadores para n\'umeros aleatorios, los que se basan en fen\'omenos f\'isicos tales como el ruido atmosf\'erico, que tiene un alto grado de entrop\'ia (ya que no se conocen las condiciones iniciales que generan estos ruidos), estos son denominados generadores de verdaderos n\'umeros aleatorios (TRNG),en contrariedad con este m\'etodo los generadores pseudoaleatorios (PRNG) \cite{b1}, necesitan de un estado inicial para generar los n\'umeros mediante una secuencia algor\'itmica, por esto carecen de entrop\'ia, es decir, si se conoce el estado inicial es muy probable que se pueda recalcular una secuencia de n\'umeros, es por esto que un buen generador pseudoaleatorio debe de incorporar alg\'un grado de entrop\'ia dentro de su ecuaci\'on,por lo tanto este tipo de t\'ecnica debe de incluir alguna complejidad adicional.\\
En este proyecto utilizaremos m\'etodos de congruencia lineal y no lineal\cite{b2} para la generaci\'on de nuestros n\'umeros pseudoaleatorios y compararemos los datos obtenidos para cada uno de estos algoritmos.
\section{Estado del arte}

\begin{itemize}
\item articulo 1


\item articulo 2


\item articulo 3

\end{itemize}


\section{Dise\~no del experimento}
\begin{itemize}

\item 
\end{itemize}



\section*{Acknowledgment}



\begin{thebibliography}{00}
\bibitem{b1} David DiCarlo (2012), Random Number Generation: Types and Techniques, [archivo PDF]. Disponible en: https://pdfs.semanticscholar.org
\bibitem{b2} James E. Gentle, Random Number Generation and Monte Carlo Methods, 2da ed., Estados Unidos, 2005, pp.11--38.

\end{thebibliography}
\vspace{12pt}

\end{document}
