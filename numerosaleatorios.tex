\documentclass[conference,a4paper]{IEEEtran}
\IEEEoverridecommandlockouts
\usepackage{cite}
\usepackage{amsmath,amssymb,amsfonts}
\usepackage{algorithmic}
\usepackage{graphicx}
\usepackage{textcomp}
\usepackage{xcolor}
\usepackage{graphicx}
\graphicspath{ {/home/junior/probabilidades/proyecto/repoclone3} }
\def\BibTeX{{\rm B\kern-.05em{\sc i\kern-.025em b}\kern-.08em
    T\kern-.1667em\lower.7ex\hbox{E}\kern-.125emX}}
\begin{document}
	
\title{GENERACI\'ON DE N\'UMEROS ALEATORIOS CON DISTRIBUCI\'ON NORMAL SIMULADO EN EL LENGUAJE PYTHON}
\author{\IEEEauthorblockN{Castillo Flores Junior \IEEEauthorrefmark{1}, Calapuja Apaza Luis \IEEEauthorrefmark{}, Nalvarte Yantas Kevin\IEEEauthorrefmark{}, Huanca Huanca Valerio\IEEEauthorrefmark{}}
\IEEEauthorblockA{Universidad Nacional de Ingenier\'ia\\
Lima, Per\'u\\
e-mail:\IEEEauthorrefmark{1}juniorcastillon6@gmail.com,
\IEEEauthorrefmark{}lcalapujaa@uni.pe,
\IEEEauthorrefmark{}kevinnalvarte@hotmail.com,
\IEEEauthorrefmark{}lhuancah@uni.pe}}
% make the title area
\maketitle
	
\begin{abstract}
%%aqui va el abstract
El objetivo de este art\'iculo es mostrar como se generan los n\'umeros aleatorios por computadora, y dar una visi\'on de cu\'ales son los algoritmos m\'as adecuados para generarlos. En este art\'iculo se implementa el m\'etodo de congruencia lineal mixta que genera n\'umeros pseudoaleatorios con la misma probabilidad de aparecer en una secuencia dada, estos n\'umeros se trabajan para obtener una distribuci\'on normal.\\
\end{abstract}
\begin{IEEEkeywords}
%%palabras claves 
Distribuci\'on normal, congruencia lineal, pseudo-aleatoriedad , estoc\'astico, 
\end{IEEEkeywords}

\section{\textbf{Introducci\'on}}
Antes de la aparici\'on de las computadoras y su capacidad de c\'alculo en procesos estoc\'asticos, existieron diferentes m\'etodos para la generaci\'on de n\'umeros aleatorios, estos estaban basados en procedimientos mec\'anicos que generaban enormes tablas de n\'umeros (Tippett, Kendall y Babbington, Rand Corporation, etc). M\'as adelante al rededor de los an\~os 40 del siglo pasado aparecen t\'ecnicas matem\'aticas para la simulaci\'on de variables aleatorias y procesos no deterministas, dicho m\'etodo lleva por nombre Monte Carlo.\\
Por otro lado existen dos tipos de generadores para n\'umeros aleatorios, los que se basan en fen\'omenos f\'isicos tales como el ruido atmosf\'erico, que tiene un alto grado de entrop\'ia (ya que no se conocen las condiciones iniciales que generan estos ruidos), estos son denominados generadores de verdaderos n\'umeros aleatorios (TRNG), en contrariedad con este m\'etodo los generadores pseudoaleatorios (PRNG), necesitan de un estado inicial para generar los n\'umeros mediante una secuencia algor\'itmica, por esto carecen de entrop\'ia, un buen generador pseudoaleatorio debe de incorporar alg\'un grado de entrop\'ia dentro de su ecuaci\'on, por lo tanto este tipo de t\'ecnica debe de incluir alguna complejidad adicional, algunas de estas t\'ecnicas se utilizan para asegurar que los n\'umeros sean fiables para su uso en criptograf\'ia.\\
En este proyecto utilizaremos m\'etodos de congruencia lineal simulados en lenguaje Python para la generaci\'on de n\'umeros pseudoaleatorios y compararemos los datos obtenidos para cada uno de estos algoritmos.
\section{\textbf{Estado del arte}}
En el art\'iculo Random Number Generation: Types and Techniques \cite{b1} se describe dos m\'etodos de generaci\'on de n\'umeros aleatorios, la primera se basa en tomar como modelo fen\'omenos del entorno f\'isico cuyos patrones no son aleatorios, hay quienes discrepan sobre esta aleatoriedad, ya que para que se d\'e un fen\'omeno debe de existir efectos que lo causen, si se llegara a conocer las condiciones iniciales que causan estos fen\'omenos, estos dejar\'ian de ser no deterministas, sin embargo, esto es casi imposible ya que los factores causantes de un fen\'omeno son pr\'acticamente infinitos, esto se puede describir como un efecto mariposa, es decir, que si se produjera una peque\~na perturbaci\'on en las condiciones iniciales los cambios en los fen\'omenos de la naturaleza ser\'ian enormes. Entre los fen\'omenos f\'isicos tomados como fuente de aleatoriedad tenemos: ruido atmosf\'erico, decaimiento radiactivo, lasers y circuitos osciladores.\\
Por otro lado tambi\'en existen generadores que no dependen de fen\'omenos de la realidad, a estos se los conoce como generadores pseudoaleatorios, estos generadores en principio parecen producir secuencias aleatorias para cualquiera que no conozca el valor inicial secreto. En un generador pseudoaleatorio b\'asico, el valor inicial es el \'unico factor en que se introduce la entrop\'ia al sistema. A diferencia de los verdaderos generadores de n\'umeros aleatorios (que toman la entrop\'ia de un fen\'omeno y lo transforman en n\'umeros), un generador pseudoaleatorio necesita encontrar alguna entrop\'ia para mantenerse impredecible. Las t\'acticas cl\'asicas para lograr esto incluyen tomar la hora del d\'ia, la ubicaci\'on del mouse o la actividad en el teclado, esto no nos asegura que alguien no pueda replicar una secuencia conociendo los valores iniciales, ya que un humano puede replicar estas t\'acticas. Un generador pseudoaleatorio confiable es aquel que no nos permite replicar con facilidad una secuencia de n\'umeros, este debe de tener variables que no puedan determinarse en un proceso intermedio.

El m\'etodo de congruencia lineal de Lehmer es uno de los m\'etodos m\'as conocidos para la generaci\'on de n\'umeros pseudoaleatorios, consiste en escoger convenientemente una semilla $X\textsubscript{0}\geq 0$, un multiplicador $a \geq 0$, incremento $c \geq 0$ y un modulo $m >X\textsubscript{0}$ ,$a$ y $c$ .

\begin{equation}
X_{j+1} = a.X_{j}+c.mod(m)
\end{equation}
Se debe de tener cuidado ya que la generaci\'on de aleatorios puede degenerarse, para esto se escoge convenientemente ciertos n\'umeros que generen un periodo m\'aximo de tama\~no $m$ que se describen a detalle en el art\'iculo \cite{b2}, con esto lograremos que los n\'umeros obtenidos sean igualmente probables, al aparecer como m\'inimo una vez en la secuencia.\\
Se describen m\'as m\'etodos que derivan de la congruencia lineal de Lehmer como: m\'etodo mixto de congruencias, m\'etodo multiplicativo de congruencias, generador Shift-Register, generador Lagger-Fibonnaci, generador de congruencia inversa, generador de congruencia lineal combinada.\\
Con los m\'etodos mencionados anteriormente se obtienen n\'umeros con distribuciones uniformes, estos pueden modificarse para generar otras distribuciones. Hay muchas distribuciones pero en nuestro estudio nos enfocaremos en la distribuci\'on normal, esto se describe con mas detalle en el cap\'itulo 5 del libro Random Number Generation and Monte Carlo Methods\cite{b3}. 
  
\section{\textbf{Dise\~no del experimento}}
Nuestra misi\'on ser\'a que dado como datos la media y la desviaci\'on t\'ipica, debemos de poder generar n\'umeros aleatorios de tal forma que tengan distribuci\'on normal con la media y la desviaci\'on t\'ipica dada.

La funci\'on de distribuci\'on, de la distribuci\'on normal de $N(\mu,\sigma^2)$ est\'a definida como:

$$\Phi_{\mu,\sigma^2}(z) = \dfrac {1}{\sigma\sqrt{2\pi}}
\int_{-\infty}^{z} e^{-\dfrac{(t-\mu)^2}{2\sigma^2}} dt, z \in {\rm I\!R} $$

Y su funci\'on densidad de probabilidad esta dada por:
$$p(z) = \dfrac {1}{\sigma\sqrt{2\pi}}
e^{-\dfrac{(z-\mu)^2}{2\sigma^2}} , z \in {\rm I\!R} $$

Esta funci\'on nos resulta en forma de campana, y nos muestra una distribuci\'on sim\'etrica, la media, la mediana y la moda toman el mismo valor igual a $\mu$ . Esta distribuci\'on es bastante usada, debido a que permite modelar numerosos fen\'omenos naturales, sociales y psicol\'ogicos.

Entonces notemos que, si hacemos $\mu=0,\sigma^2=1$ ,obtenemos $N(0,1)$
la cual es denominada distribuci\'on normal est\'andar, 

$$p(x) = \dfrac {1}{\sqrt{2\pi}}
e^{-\dfrac{x^2}{2}} dt, x \in {\rm I\!R} $$
Debemos de notar que haciendo un cambio de variable en la funci\'on de distribuci\'on normal est\'andar $$z=\dfrac{x-\mu}{\sigma}$$ podemos obtener nuevamente la distribuci\'on normal $N(\mu,\sigma^2)$ 

$$p(z) = \dfrac {1}{\sigma\sqrt{2\pi}}
e^{-\dfrac{(z-\mu)^2}{2\sigma^2}} , z \in {\rm I\!R} $$

Por lo tanto, haremos el experimento usando la distribuci\'on normal est\'andar $N(0,1)$.

Esto quiere decir que los n\'umeros aleatorios deben generarse teniendo en cuenta que para cada tramo existe una probabilidad diferente, que depende de que tan cerca o lejos de la media se encuentre.

Entonces siguiendo con el experimento, podemos usar una distribuci\'on uniforme para construir nuestra distribuci\'on normal est\'andar. Encontramos material para esto en el cap\'itulo 5 del libro Random Number Generation and Monte Carlo Methods citado anteriormente.

Por lo tanto, en primer lugar tendremos que conseguir un algoritmo eficiente para la generaci\'on de n\'umeros aleatorios distribuidos uniformemente. 

Existen varios m\'etodos para conseguir esto, pero nosotros implementaremos el algoritmo m\'as usado y recomendado en las bibliograf\'ias, por su simpleza, facilidad y eficacia, el cual es: el m\'etodo de generaci\'on de n\'umeros pseudoaleatorios usando generadores de congruencia lineal mixta, el cual se basa en el m\'etodo de congruencia lineal de Lehmer.

El m\'etodo de congruencia lineal de Lehmer pasa a llamarse m\'etodo mixto de congruencias cuando $c \not= 0$ en la siguiente sucesi\'on

\begin{equation}
X_{j+1} = a.X_{j}+c.mod(m)
\end{equation}


Entonces el dise\~no del experimento se realizar\'a en tres partes bien diferenciadas:\\

\begin{enumerate}
	\item Implementar de manera \'optima un algoritmo para conseguir la generaci\'on de n\'umeros aleatorios en distribuci\'on uniforme. 
	\item Implementar la modificaci\'on y/o el uso del algoritmo de generaci\'on de n\'umeros aleatorios en distribuci\'on uniforme para obtener un generador de n\'umeros aleatorios en distribuci\'on normal est\'andar.
	\item Hacer el cambio de variable y usar el algoritmo de generaci\'on de n\'umeros aleatorios en distribuci\'on normal est\'andar para obtener un algoritmo de generaci\'on de n\'umeros aleatorios en distribuci\'on normal el cual tendr\'a como argumentos adicionales la media ($\mu$) y la desviaci\'on t\'ipica ($\sigma^2$).
	
\end{enumerate}
Tambien utilizaremos la t\'ecnica de Box-Muller, que transforma dos distribuciones uniformes en una distribuci\'on normal bivariada.\\
El m\'etodo consiste en los siguientes pasos:
\begin{enumerate}
	\item Se generan dos n\'umeros aleatorios $r_{1}$ y $r_{2}$ , $U(0,1)$
	\item Se transforman en dos variables aleatorias normales, cada	una con media 0 y varianza 1, usando transformaciones directas:\\
	\begin{equation}
	z_{1}=(-2\ln r_{1})^{1/2} \sin (2\pi r_{2})
	\end{equation}
	\begin{equation}
	z_{2}=(-2\ln r_{1})^{1/2} \cos (2\pi r_{2})
	\end{equation}
	\item Se calculan las variables aleatorias normales $x_{1}$ y $x_{2}$ de la siguiente forma:\\
	\begin{equation}
	x_{1}=z_{1}\sigma + \mu
	\end{equation}
	\begin{equation}
	x_{2}=z_{2}\sigma + \mu
	\end{equation}
\end{enumerate}
\section{\textbf{Experimentos y resultados}}
A partir del m\'etodo de congruencia lineal, tomando como par\'ametros $a = 7^{5}$ , $c = 630360016$ y $m = 2^{31}-1$ ,elegidos convenientemente, generamos varias muestras pseudoaleatorias, dichas muestras se dividieron entre el m\'odulo para uniformizarlas  $U$ $(0,1)$ , para luego aplicar el teorema del l\'imite central que transforma esta distribuci\'on en una distribuci\'on normal seg\'un:\\
\begin{equation}
 z = \frac{\sum_{i=1}^{n}r_{i}  - n\mu}{(n\sigma^{2})^{1/2}}
\end{equation}
Donde $r_{i}$ son los valores asociados a las variables aleatorias iid uniformes $U_{i}$ en (0,1), asi obtenemos para n muestras, n valores para z , estos valores tiene una distribuci\'on  $N(0,1)$ como se obtuvo en el programa dibujando un histograma de frecuencias:\\
\includegraphics[scale=0.55]{normal}\\
Para el m\'etodo de Box-Muller utilizamos el mismo generador de congruencia lineal, y obtuvimos 2 variables aleatorias distribuidas normalmente, aplicando las ecuaciones (3) y (4):\\
\includegraphics[scale=0.55]{z1}  
\includegraphics[scale=0.55]{z2}
Si graficamos esta distribuci\'on obtenida con el m\'etodo de box-muller como una proyecci\'on en 2 dimensiones, obtenemos:\\
\includegraphics[scale=0.5]{bivariada}
\section{Discusi\'on}
Se debe de tener cuidado al momento de elegir los par\'ametros para el algoritmo de congruencia lineal, ya que debe de cumplir con ciertas condiciones para que no se degenere, es decir, su periodo m\'aximo llegue a m-1.\\

En el m\'etodo de convoluci\'on (teorema del limite central)  nos arroja valores entre -1 y 1 de manera aleatoria, cuya distribuci\'on es normal cuando la cantidad de muestras es mayor que 5, nostros tomamos para 1000 muestras iid a partir de nuestro generador pseudoaleatorio.\\

El m\'etodo de box-muller es un m\'etodo directo de transformar 2 variables distribuidas uniformemente en U(0,1)
en 2 variables aleatorias distribuidas normalmente. que en coordenadas polares nos da una distribucion bivariada normal. Nosotros graficamos la proyecci\'on en 2 dimensiones, y podemos observar que los datos se juntan hacia el centro, de la circunferencia, es decir, hay mas datos cerca de la media  y a medida que nos alejamos del centro los puntos se dispersan.\\

\section{Conclusiones}
Los m\'etodos utilizados para la transformaci\'on de una distribucion uniforme a una con distribuci\'on normal son confialbes, siempre y cuando las muestras uniformes sean iid (independientes e identicamente distribuidas).\\

Se debe de elegir buenos parametros para el generador de congruencia lineal ,que cumplan las condiciones mencionadas, para que nuestros numeros pseudoaleatorios no degeneren.\\
\section*{Acknowledgment}



\begin{thebibliography}{00}
\bibitem{b1} David DiCarlo (2012), Random Number Generation: Types and Techniques, [archivo PDF]. Disponible en: https://pdfs.semanticscholar.org
\bibitem{b2} Alfonso Mancilla (2000), N\'umeros Aleatorios ,[archivo PDF]. Disponible en: http://ciruelo.uninorte.edu.co/pdf/ingenieria\_desarrollo/
\bibitem{b3} James E. Gentle, Random Number Generation and Monte Carlo Methods, 2da ed., Estados Unidos, 2005, pp.165--217.

\end{thebibliography}
\vspace{12pt}

\end{document}
	